\documentclass[a4paper]{arrowhead}

\usepackage[yyyymmdd]{datetime}
\usepackage{etoolbox}
\usepackage[utf8]{inputenc}
\usepackage{multirow}

\renewcommand{\dateseparator}{-}

\setlength{\parskip}{1em}

%% Special references
\newcommand{\fref}[1]{{\textcolor{ArrowheadBlue}{\hyperref[sec:functions:#1]{#1}}}}
\newcommand{\mref}[1]{{\textcolor{ArrowheadPurple}{\hyperref[sec:model:#1]{#1}}}}
\newcommand{\pdef}[1]{{\textcolor{ArrowheadGrey}{#1\label{sec:model:primitives:#1}\label{sec:model:primitives:#1s}\label{sec:model:primitives:#1es}}}}
\newcommand{\pref}[1]{{\textcolor{ArrowheadGrey}{\hyperref[sec:model:primitives:#1]{#1}}}}

\newrobustcmd\fsubsection[3]{
  \addtocounter{subsection}{1}
  \addcontentsline{toc}{subsection}{\protect\numberline{\thesubsection}interface \textcolor{ArrowheadBlue}{#1}}
  \renewcommand*{\do}[1]{\rref{##1},\ }
  \subsection*{
    \thesubsection\quad
    interface
    \textcolor{ArrowheadBlue}{#1}
    (\notblank{#2}{\mref{#2}}{})
    \notblank{#3}{: \mref{#3}}{}
  }
  \label{sec:functions:#1}
}
\newrobustcmd\msubsection[2]{
  \addtocounter{subsection}{1}
  \addcontentsline{toc}{subsection}{\protect\numberline{\thesubsection}#1 \textcolor{ArrowheadPurple}{#2}}
  \subsection*{\thesubsection\quad#1 \textcolor{ArrowheadPurple}{#2}}
  \label{sec:model:#2} \label{sec:model:#2s} \label{sec:model:#2es}
}
\newrobustcmd\msubsubsection[3]{
  \addtocounter{subsubsection}{1}
  \addcontentsline{toc}{subsubsection}{\protect\numberline{\thesubsubsection}#1 \textcolor{ArrowheadPurple}{#2}}
  \subsubsection*{\thesubsubsection\quad#1 \textcolor{ArrowheadPurple}{#2}}
  \label{sec:model:#2} \label{sec:model:#2s}
}
%%

\begin{document}

%% Arrowhead Document Properties
\ArrowheadTitle{token-generation} % XXX = ServiceName 
\ArrowheadServiceID{token-generation} % ID name of service
\ArrowheadType{Service Description}
\ArrowheadTypeShort{SD}
\ArrowheadVersion{4.6.0} % Arrowhead version X.Y.Z, e..g. 4.4.1
\ArrowheadDate{\today}
\ArrowheadAuthor{Tamás Bordi} % Corresponding author e.g. Jerker Delsing
\ArrowheadStatus{RELEASE} % e..g. RELEASE, RELEASE CONDIDATE, PROTOTYPE
\ArrowheadContact{tbordi@aitia.ai} % Email of corresponding author
\ArrowheadFooter{\href{www.arrowhead.eu}{www.arrowhead.eu}}
\ArrowheadSetup
%%

%% Front Page
\begin{center}
  \vspace*{1cm}
  \huge{\arrowtitle}

  \vspace*{0.2cm}
  \LARGE{\arrowtype}
  \vspace*{1cm}

  %\Large{Service ID: \textit{"\arrowid"}}
  \vspace*{\fill}

  % Front Page Image
  %\includegraphics{figures/TODO}

  \vspace*{1cm}
  \vspace*{\fill}

  % Front Page Abstract
  \begin{abstract}
    This document provides service description for the \textbf{token-generation} service. 
  \end{abstract}

  \vspace*{1cm}

%   \scriptsize
%   \begin{tabularx}{\textwidth}{l X}
%     \raisebox{-0.5\height}{\includegraphics[width=2cm]{figures/artemis_logo}} & {ARTEMIS Innovation Pilot Project: Arrowhead\newline
%     THEME [SP1-JTI-ARTEMIS-2012-AIPP4 SP1-JTI-ARTEMIS-2012-AIPP6]\newline
%     [Production and Energy System Automation Intelligent-Built environment and urban infrastructure for sustainable and friendly cities]}
%   \end{tabularx}
%   \vspace*{-0.2cm}
 \end{center}

\newpage
%%

%% Table of Contents
\tableofcontents
\newpage
%%

\section{Overview}
\label{sec:overview}
This document describes the \textbf{token-generation} service, which enables session control within and between local clouds. The purpose of this service is to generate access tokens for a consumer to a provider with the content of service consumption session related data.

The rest of this document is organized as follows.
In Section \ref{sec:functions}, we describe the abstract message functions provided by the service.
In Section \ref{sec:model}, we end the document by presenting the data types used by the mentioned functions.

\subsection{How This Service Is Meant to Be Used}
Primarily the Orchestrator Core System should consume this service during the orchestration process, when a registered provider - which could be fulfill the consumer's orchestration request - requires \texttt{TOKEN} level security.

The generated token has to meet the following criteria:
\begin{itemize}
    \item It must contain the issuer system's name.
    \item It must contain an issued at timestamp.
    \item It must contain a valid from time 
    \item It could contain an expiration time.
    \item It must contain a consumer identifier.
    \item It must contain a service identifier.
    \item It must contain a interface identifier.
    \item It must be signed by Authorization Core System,
    \item It must be encrypted with the provider's public key. 
\end{itemize}

\subsection{Important Delimitations}
\label{sec:delimitations}

The access token generation is possible only when the hosting system is in secure mode.

Supported token types:

\begin{itemize}
    \item JSON Web Token (JWT) with signing algorithm RSA using SHA-512 (\texttt{"RS512"}), with content encryption algorithm AES-256-CBC-HMAC-SHA-512 (\texttt{"A256CBC-HS512"}) and with key encryption algorithm RSA-OAEP-256 (\texttt{"RSA-OAEP-256"}). \color{blue} \href{https://jwt.io/}{More details and libraries.} \color{black}
\end{itemize}

\subsection{Access policy}
\label{sec:accesspolicy}

This service is available only for the Orchestrator and the Choreographer Core Systems.

\newpage

\section{Service Interface}
\label{sec:functions}

This section describes the interfaces to the service. The \textbf{token-generation} service is used to generate access tokens. The various parameters are representing the necessary system and service input information.
In particular, each subsection names an interface, an input type and an output type, in that order.
The input type is named inside parentheses, while the output type is preceded by a colon.
Input and output types are only denoted when accepted or returned, respectively, by the interface in question. All abstract data types named in this section are defined in Section 3.

The following interfaces are available.

\fsubsection{HTTP/TLS/JSON}{TokenGenerationRequest}{TokenGenerationRes-ponse}

\begin{table}[ht!]
  \centering
  \begin{tabular}{|l|l|l|l|}
    \rowcolor{gray!33} Profile ype & Type & Version \\ \hline
    Transfer protocol & HTTP & 1.1 \\ \hline
    Data encryption & TLS & 1.3 \\ \hline
    Encoding & JSON & RFC 8259 \cite{rfc8259} \\ \hline
    Compression & N/A & - \\ \hline
  \end{tabular}
  \caption{HTTP/TLS/JSON communication details.}
  \label{tab:comunication_semantics_profile}
\end{table}

\clearpage

\section{Information Model}
\label{sec:model}

Here, all data objects that can be part of the \textbf{token-generation} service
provides to the hosting System are listed in alphabetic order.
Note that each subsection, which describes one type of object, begins with the \textit{struct} keyword, which is used to denote a collection of named fields, each with its own data type.
As a complement to the explicitly defined types in this section, there is also a list of implicit primitive types in Section \ref{sec:model:primitives}, which are used to represent things like hashes and identifiers.

\msubsection{struct}{TokenGenerationRequest}
\label{sec:model:TokenGenerationRequest}
 
\begin{table}[ht!]
\begin{tabularx}{\textwidth}{| p{3cm} | p{6cm} | p{2cm} | X |} \hline
\rowcolor{gray!33} Field & Type & Mandatory & Description \\ \hline
consumer & \hyperref[sec:model:SystemDescriptor]{SystemDescriptor} & yes & Descriptor of the consumer system. \\ \hline
consumerCloud & \hyperref[sec:model:CloudDescriptor]{CloudDescriptor} & no &  Descriptor of the consumer cloud. \\ \hline
providers &  \pref{List}$<$\hyperref[sec:model:TokenGenerationDescriptor]{TokenGenerationDescriptor}$>$ & yes & Array of token generation descriptors. \\ \hline
service &\pref{Name} & yes & Identifier of the service. \\ \hline
\end{tabularx}
\end{table}

\msubsection{struct}{SystemDescriptor}
\label{sec:model:SystemDescriptor}

\begin{table}[ht!]
\begin{tabularx}{\textwidth}{| p{3cm} | p{3cm} | p{2cm} | X |} \hline
\rowcolor{gray!33} Field & Type & Mandatory & Description \\ \hline
address & \pref{Address} & yes & Network address. \\ \hline
authenticationInfo & \pref{String} & yes in case of providers & Public key of the client certificate. \\ \hline
metadata & \hyperref[sec:model:Metadata]{Metadata} & no & Metadata. \\ \hline
port & \pref{PortNumber} & yes & Port of the system. \\ \hline
systemName &\pref{Name} & yes & Name of the system. \\ \hline
\end{tabularx}
\end{table}

\msubsection{struct}{Metadata}
\label{sec:model:Metadata}

An \pref{Object} which maps \pref{String} key-value pairs.

\msubsection{struct}{CloudDescriptor}
\label{sec:model:CloudDescriptor}

\begin{table}[ht!]
\begin{tabularx}{\textwidth}{| p{3cm} | p{3cm} | p{2cm} | X |} \hline
\rowcolor{gray!33} Field & Type & Mandatory & Description. \\ \hline
name & \pref{Name} & yes & Name of the cloud. \\ \hline
operator & \pref{Name} & yes & Name of the cloud operator. \\ \hline
\end{tabularx}
\end{table}

\clearpage

\msubsection{struct}{TokenGenerationDescriptor}
\label{sec:model:TokenGenerationDescriptor}

\begin{table}[ht!]
\begin{tabularx}{\textwidth}{| p{3cm} | p{6cm} | p{2cm} | X |} \hline
\rowcolor{gray!33} Field & Type & Mandatory & Description \\ \hline
provider & \hyperref[sec:model:SystemDescriptor]{SystemDescriptor} & yes & Descriptor of the provider system. \\ \hline
interfaces & \pref{List}$<$\pref{Name}$>$ & yes & List of interface names. \\ \hline
tokenDuration &  \pref{Number} & no & Validity period of the token in se\-conds. \\ \hline
\end{tabularx}
\end{table}

\msubsection{struct}{TokenGenerationResponse}
\label{sec:model:TokenGenerationResponse}
 
\begin{table}[ht!]
\begin{tabularx}{\textwidth}{| p{4.25cm} | p{3.5cm} | X |} \hline
\rowcolor{gray!33} Field & Type      & Description \\ \hline
tokenData & \pref{List}$<$\hyperref[sec:model:TokenData]{TokenData}$>$ & List of token data descriptors. \\ \hline
\end{tabularx}
\end{table}

\msubsection{struct}{TokenData}
\label{sec:model:TokenData}

\begin{table}[ht!]
\begin{tabularx}{\textwidth}{| p{4.25cm} | p{3.5cm} | X |} \hline
\rowcolor{gray!33} Field & Type      & Description \\ \hline
providerAddress & \pref{Address} & Network address of the system. \\ \hline
providerPort & \pref{PortNumber} & Port of the system. \\ \hline
providerName & \pref{Name} & Name of the system. \\ \hline
tokens &  \pref{Map}$<$\pref{Interface},\hyperref[sec:model:Token]{Token}$>$ & Interface-token pairs. \\ \hline 

\end{tabularx}
\end{table}

\msubsection{struct}{Token}
\label{sec:model:Token}

An encrypted \pref{String} format credential which holds system, cloud, service and session information.

\subsection{Primitives}
\label{sec:model:primitives}

Types and structures mentioned throughout this document that are assumed to be available to implementations of this service.
The concrete interpretations of each of these types and structures must be provided by any IDD document claiming to implement this service.


\begin{table}[ht!]
\begin{tabularx}{\textwidth}{| p{3cm} | X |} \hline
\rowcolor{gray!33} Type & Description \\ \hline
\pdef{Address}          & A string representation of the address. \\ \hline
\pdef{Object}           & Set of primitives and possible further objects. \\ \hline
\pdef{Interface}        & Any suitable type chosen by the implementor of service \\ \hline
\pdef{List}$<$A$>$      & An \textit{array} of a known number of items, each having type A. \\ \hline
\pdef{Map}$<$A,B$>$    & An \textit{object} which maps key-value pairs. Each key having A type and each value having B type. \\ \hline
\pdef{Name}             & A string identifier that is intended to be both human and machine-readable. \\ \hline
\pdef{Number}           & Decimal number \\ \hline
\pdef{PortNumber}       & A \pref{Number} between 0 and 65535. \\ \hline
\pdef{String}           & A chain of characters. \\ \hline
\end{tabularx}
\end{table}

\newpage

\bibliographystyle{IEEEtran}
\bibliography{bibliography}

\newpage

\section{Revision History}
\subsection{Amendments}

\noindent\begin{tabularx}{\textwidth}{| p{1cm} | p{3cm} | p{2cm} | X | p{4cm} |} \hline
\rowcolor{gray!33} No. & Date & Version & Subject of Amendments & Author \\ \hline

1 & YYYY-MM-DD & \arrowversion & & Xxx Yyy \\ \hline
\end{tabularx}

\subsection{Quality Assurance}

\noindent\begin{tabularx}{\textwidth}{| p{1cm} | p{3cm} | p{2cm} | X |} \hline
\rowcolor{gray!33} No. & Date & Version & Approved by \\ \hline

1 & YYYY-MM-DD & \arrowversion  &  \\ \hline

\end{tabularx}

\end{document}