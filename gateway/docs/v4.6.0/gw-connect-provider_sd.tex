\documentclass[a4paper]{arrowhead}

\usepackage[yyyymmdd]{datetime}
\usepackage{etoolbox}
\usepackage[utf8]{inputenc}
\usepackage{multirow}

\renewcommand{\dateseparator}{-}

\setlength{\parskip}{1em}

%% Special references
\newcommand{\fref}[1]{{\textcolor{ArrowheadBlue}{\hyperref[sec:functions:#1]{#1}}}}
\newcommand{\mref}[1]{{\textcolor{ArrowheadPurple}{\hyperref[sec:model:#1]{#1}}}}
\newcommand{\pdef}[1]{{\textcolor{ArrowheadGrey}{#1\label{sec:model:primitives:#1}\label{sec:model:primitives:#1s}\label{sec:model:primitives:#1es}}}}
\newcommand{\pref}[1]{{\textcolor{ArrowheadGrey}{\hyperref[sec:model:primitives:#1]{#1}}}}

\newrobustcmd\fsubsection[3]{
  \addtocounter{subsection}{1}
  \addcontentsline{toc}{subsection}{\protect\numberline{\thesubsection}interface \textcolor{ArrowheadBlue}{#1}}
  \renewcommand*{\do}[1]{\rref{##1},\ }
  \subsection*{
    \thesubsection\quad
    interface
    \textcolor{ArrowheadBlue}{#1}
    (\notblank{#2}{\mref{#2}}{})
    \notblank{#3}{: \mref{#3}}{}
  }
  \label{sec:functions:#1}
}
\newrobustcmd\msubsection[2]{
  \addtocounter{subsection}{1}
  \addcontentsline{toc}{subsection}{\protect\numberline{\thesubsection}#1 \textcolor{ArrowheadPurple}{#2}}
  \subsection*{\thesubsection\quad#1 \textcolor{ArrowheadPurple}{#2}}
  \label{sec:model:#2} \label{sec:model:#2s} \label{sec:model:#2es}
}
\newrobustcmd\msubsubsection[3]{
  \addtocounter{subsubsection}{1}
  \addcontentsline{toc}{subsubsection}{\protect\numberline{\thesubsubsection}#1 \textcolor{ArrowheadPurple}{#2}}
  \subsubsection*{\thesubsubsection\quad#1 \textcolor{ArrowheadPurple}{#2}}
  \label{sec:model:#2} \label{sec:model:#2s}
}
%%

\begin{document}

%% Arrowhead Document Properties
\ArrowheadTitle{gw-connect-provider} % XXX = ServiceName 
\ArrowheadServiceID{gw-connect-provider} % ID name of service
\ArrowheadType{Service Description}
\ArrowheadTypeShort{SD}
\ArrowheadVersion{4.6.0} % Arrowhead version X.Y.Z, e..g. 4.4.1
\ArrowheadDate{\today}
\ArrowheadAuthor{Rajmund Bocsi} % Corresponding author e.g. Jerker Delsing
\ArrowheadStatus{RELEASE} % e..g. RELEASE, RELEASE CONDIDATE, PROTOTYPE
\ArrowheadContact{rbocsi@aitia.ai} % Email of corresponding author
\ArrowheadFooter{\href{www.arrowhead.eu}{www.arrowhead.eu}}
\ArrowheadSetup
%%

%% Front Page
\begin{center}
  \vspace*{1cm}
  \huge{\arrowtitle}

  \vspace*{0.2cm}
  \LARGE{\arrowtype}
  \vspace*{1cm}

  %\Large{Service ID: \textit{"\arrowid"}}
  \vspace*{\fill}

  % Front Page Image
  %\includegraphics{figures/TODO}

  \vspace*{1cm}
  \vspace*{\fill}

  % Front Page Abstract
  \begin{abstract}
    This document provides service description for the \textbf{gw-connect-provider} service. 
  \end{abstract}

  \vspace*{1cm}

%   \scriptsize
%   \begin{tabularx}{\textwidth}{l X}
%     \raisebox{-0.5\height}{\includegraphics[width=2cm]{figures/artemis_logo}} & {ARTEMIS Innovation Pilot Project: Arrowhead\newline
%     THEME [SP1-JTI-ARTEMIS-2012-AIPP4 SP1-JTI-ARTEMIS-2012-AIPP6]\newline
%     [Production and Energy System Automation Intelligent-Built environment and urban infrastructure for sustainable and friendly cities]}
%   \end{tabularx}
%   \vspace*{-0.2cm}
 \end{center}

\newpage
%%

%% Table of Contents
\tableofcontents
\newpage
%%

\section{Overview}
\label{sec:overview}
This document describes the \textbf{gw-connect-provider} service which enables the Gatekeeper to create a proxy that can consume a specific provider's service in the local cloud in the name of a consumer beyond the cloud boundaries. The proxy connects to its consumer side equivalent via a relay. Requests comes through the relay connection are forwarded to the provider and the provider's response is sent back to the consumer side via the relay connection.

The rest of this document is organized as follows.
In Section \ref{sec:functions}, we describe the abstract message functions provided by the service.
In Section \ref{sec:model}, we end the document by presenting the data types used by the mentioned functions.

\subsection{How This Service Is Meant to Be Used}
The Gatekeeper should consume the Service Registry Core System’s \textbf{query} service to get information about the \textbf{gw-connect-provider} service. Using this information the system can create a proxy for the inter-cloud service consumption.

\subsection{Important Delimitations}
\label{sec:delimitations}

It only supports services with secured interfaces that use TCP-based protocols.

\subsection{Access policy}
\label{sec:accesspolicy}

This service is only available for the Gatekeeper Core System.

\newpage

\section{Service Interface}
\label{sec:functions}

This section describes the interfaces to the service. The \textbf{gw-connect-provider} service is used to create a proxy that provider access to a specific provider's service to a foreign consumer. The various parameters are representing the necessary cloud, system and service input information, etc.
In particular, each subsection names an interface, an input type and an output type, in that order.
The input type is named inside parentheses, while the output type is preceded by a colon.
Input and output types are only denoted when accepted or returned, respectively, by the interface in question. All abstract data types named in this section are defined in Section 3.

The following interfaces are available.

\fsubsection{HTTP/TLS/JSON}{ConnectionResponse}{ConnectionRequest}

\begin{table}[ht!]
  \centering
  \begin{tabular}{|l|l|l|l|}
    \rowcolor{gray!33} Profile type & Type & Version \\ \hline
    Transfer protocol & HTTP & 1.1 \\ \hline
    Data encryption & TLS & 1.3 \\ \hline
    Encoding & JSON & RFC 8259 \cite{rfc8259} \\ \hline
    Compression & N/A & - \\ \hline
  \end{tabular}
  \caption{HTTP/TLS/JSON communication details.}
  \label{tab:comunication_semantics_profile}
\end{table}

\clearpage

\section{Information Model}
\label{sec:model}

Here, all data objects that can be part of the \textbf{gw-connect-provider} service provides to the hosting System are listed in alphabetic order. Note that each subsection, which describes one type of object, begins with the \textit{struct} keyword, which is used to denote a collection of named fields, each with its own data type. As a complement to the explicitly defined types in this section, there is also a list of implicit primitive types in Section \ref{sec:model:primitives}, which are used to represent things like hashes and identifiers.

\msubsection{struct}{ConnectionRequest}
\label{sec:model:ConnectionRequest}
 
\begin{table}[ht!]
\begin{tabularx}{\textwidth}{| p{3.6cm} | p{3cm} | p{2cm} | X |} \hline
\rowcolor{gray!33} Field & Type & Mandatory & Description \\ \hline
consumer & \hyperref[sec:model:System]{System} & yes & Information about the system that originally requests the service. \\ \hline
consumerCloud & \hyperref[sec:model:Cloud]{Cloud} & yes & Information about the cloud where the consumer is located. \\ \hline
consumerGWPublicKey & \pref{String} & yes & The X.509 public key of the consumer side Gateway.  \\ \hline
provider & \hyperref[sec:model:System]{System} & yes & Information about the selected provider system. \\ \hline
providerCloud & \hyperref[sec:model:Cloud]{Cloud} & yes & Information about the cloud where the provider is located. \\ \hline
relay & \hyperref[sec:model:Relay]{Relay} & yes & Information about the relay that will handle the communication between the two participating clouds. \\ \hline
serviceDefinition & \pref{String} & yes & The name of the service that the original consumer want to use. \\ \hline

\end{tabularx}
\end{table}

\msubsection{struct}{System}
\label{sec:model:System}

\begin{table}[ht!]
\begin{tabularx}{\textwidth}{| p{4cm} | p{4cm} | p{2cm} | X |} \hline
\rowcolor{gray!33} Field & Type & Mandatory & Description \\ \hline

address &\pref{Address} & yes & Network address of the system. \\ \hline
authenticationInfo &\pref{String} & no & X.509 public key of the system. \\ \hline
metadata &\hyperref[sec:model:Metadata]{Metadata} & no & Additional information about the system. \\ \hline
port &\pref{PortNumber} & yes & Port of the system. \\ \hline
systemName &\pref{Name} & yes & Name of the system. \\ \hline
\end{tabularx}
\end{table}

\msubsection{struct}{Metadata}
\label{sec:model:Metadata}

An \pref{Object} which maps \pref{String} key-value pairs.

\clearpage

\msubsection{struct}{Cloud}
\label{sec:model:Cloud}

\begin{table}[ht!]
\begin{tabularx}{\textwidth}{| p{4cm} | p{4cm} | p{2cm} | X |} \hline
\rowcolor{gray!33} Field & Type & Mandatory & Description \\ \hline

name &\pref{Name} & yes & Name of the cloud. \\ \hline
operator &\pref{Name} & yes & Operator of the cloud. \\ \hline
\end{tabularx}
\end{table}

\msubsection{struct}{Relay}
\label{sec:model:Relay}

\begin{table}[ht!]
\begin{tabularx}{\textwidth}{| p{4cm} | p{4cm} | p{2cm} | X |} \hline
\rowcolor{gray!33} Field & Type & Mandatory & Description \\ \hline

address &\pref{Address} & yes & Network address of the relay. \\ \hline
port &\pref{PortNumber} & yes & Port of the relay. \\ \hline
secure &\pref{Boolean} & no & Whether the relay secure or not. \\ \hline
type &\pref{RelayType} & yes & The type of the relay. Note that \textit{GATEKEEPER\_RELAY} is not accepted here. \\ \hline
\end{tabularx}
\end{table}

\msubsection{struct}{ConnectionResponse}
\label{sec:model:ConnectionResponse}
 
\begin{table}[ht!]
\begin{tabularx}{\textwidth}{| p{3.5cm} | p{5.0cm} | X |} \hline
\rowcolor{gray!33} Field & Type      & Description \\ \hline
peerName & \pref{String} & The common name (CN) of the provider side Gateway Core System. \\ \hline
providerGWPublicKey & \pref{String} & The X.509 public key of the provider side Gateway. \\ \hline
queueId & \pref{String} & The random part of the queue names that the relay communication uses. \\ \hline
\end{tabularx}
\end{table}

\subsection{Primitives}
\label{sec:model:primitives}

Types and structures mentioned throughout this document that are assumed to be available to implementations of this service.
The concrete interpretations of each of these types and structures must be provided by any IDD document claiming to implement this service.

\begin{table}[ht!]
\begin{tabularx}{\textwidth}{| p{3cm} | X |} \hline
\rowcolor{gray!33} Type & Description \\ \hline
\pdef{Address}          & A string representation of the address. \\ \hline
\pdef{Boolean}          & One out of \texttt{true} or \texttt{false}. \\ \hline
\pdef{Object}           & Set of primitives and possible further objects. \\ \hline
\pdef{Name}             & A string identifier that is intended to be both human and machine-readable. \\ \hline
\pdef{Number}           & Decimal number \\ \hline
\pdef{PortNumber}       & A \pref{Number} between 0 and 65535. \\ \hline
\pdef{RelayType} & The purpose of the relay. Relays can be specified for general purpose, or specifically for Gatekeeper or Gateway communication. \\ \hline
\pdef{String}           & A chain of characters. \\ \hline

\end{tabularx}
\end{table}

\newpage

\bibliographystyle{IEEEtran}
\bibliography{bibliography}

\newpage

\section{Revision History}
\subsection{Amendments}

\noindent\begin{tabularx}{\textwidth}{| p{1cm} | p{3cm} | p{2cm} | X | p{4cm} |} \hline
\rowcolor{gray!33} No. & Date & Version & Subject of Amendments & Author \\ \hline

1 & YYYY-MM-DD & \arrowversion & & Xxx Yyy \\ \hline
\end{tabularx}

\subsection{Quality Assurance}

\noindent\begin{tabularx}{\textwidth}{| p{1cm} | p{3cm} | p{2cm} | X |} \hline
\rowcolor{gray!33} No. & Date & Version & Approved by \\ \hline

1 & YYYY-MM-DD & \arrowversion  &  \\ \hline

\end{tabularx}

\end{document}