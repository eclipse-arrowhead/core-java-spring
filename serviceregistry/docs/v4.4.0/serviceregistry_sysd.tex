\documentclass[a4paper]{arrowhead}

\usepackage[yyyymmdd]{datetime}
\usepackage{etoolbox}
\usepackage[utf8]{inputenc}
\usepackage{multirow}

\renewcommand{\dateseparator}{-}

\setlength{\parskip}{1em}

%% Special references
\newcommand{\fref}[1]{{\textcolor{ArrowheadBlue}{\hyperref[sec:functions:#1]{#1}}}}
\newcommand{\mref}[1]{{\textcolor{ArrowheadPurple}{\hyperref[sec:model:#1]{#1}}}}
\newcommand{\pdef}[1]{{\textcolor{ArrowheadGrey}{#1\label{sec:model:primitives:#1}\label{sec:model:primitives:#1s}\label{sec:model:primitives:#1es}}}}
\newcommand{\pref}[1]{{\textcolor{ArrowheadGrey}{\hyperref[sec:model:primitives:#1]{#1}}}}

\newrobustcmd\fsubsection[3]{
  \addtocounter{subsection}{1}
  \addcontentsline{toc}{subsection}{\protect\numberline{\thesubsection}function \textcolor{ArrowheadBlue}{#1}}
  \renewcommand*{\do}[1]{\rref{##1},\ }
  \subsection*{
    \thesubsection\quad
    operation
    \textcolor{ArrowheadBlue}{#1}
    (\notblank{#2}{\mref{#2}}{})
    \notblank{#3}{: \mref{#3}}{}
  }
  \label{sec:functions:#1}
}
\newrobustcmd\msubsection[2]{
  \addtocounter{subsection}{1}
  \addcontentsline{toc}{subsection}{\protect\numberline{\thesubsection}#1 \textcolor{ArrowheadPurple}{#2}}
  \subsection*{\thesubsection\quad#1 \textcolor{ArrowheadPurple}{#2}}
  \label{sec:model:#2} \label{sec:model:#2s} \label{sec:model:#2es}
}

\begin{document}

%% Arrowhead Document Properties
\ArrowheadTitle{Service Registry Core System}
\ArrowheadType{System Description}
\ArrowheadTypeShort{SysD}
\ArrowheadVersion{4.4.0}
\ArrowheadDate{\today}
\ArrowheadAuthor{Tamás Bordi}
\ArrowheadStatus{RELEASE}
\ArrowheadContact{tbordi@aitia.ai}
\ArrowheadFooter{\href{www.arrowhead.eu}{www.arrowhead.eu}}
\ArrowheadSetup
%%

%% Front Page
\begin{center}
  \vspace*{1cm}
  \huge{\arrowtitle}

  \vspace*{0.2cm}
  \LARGE{\arrowtype}
  \vspace*{1cm}

  %\Large{Service ID: \textit{"\arrowid"}}
  \vspace*{\fill}

  % Front Page Image
  %\includegraphics{figures/TODO}

  \vspace*{1cm}
  \vspace*{\fill}

  % Front Page Abstract
  \begin{abstract}
    This is the template for System Description (SysD document)
    according to the Eclipse Arrowehad documentation structure. 
  \end{abstract}

  \vspace*{1cm}

%   \scriptsize
%   \begin{tabularx}{\textwidth}{l X}
%     \raisebox{-0.5\height}{\includegraphics[width=2cm]{figures/artemis_logo}} & {ARTEMIS Innovation Pilot Project: Arrowhead\newline
%     THEME [SP1-JTI-ARTEMIS-2012-AIPP4 SP1-JTI-ARTEMIS-2012-AIPP6]\newline
%     [Production and Energy System Automation Intelligent-Built environment and urban infrastructure for sustainable and friendly cities]}
%   \end{tabularx}
%   \vspace*{-0.2cm}
 \end{center}

\newpage
%%

%% Table of Contents
\tableofcontents
\newpage
%%

\section{Overview}
\label{sec:overview}
\color{black}
This document describes the Service Registry Core System, which exists to enable service discovery in a Eclipse Arrowhead Local Cloud (LC). Examples of such interactions is a provider system offering some kind of Service for use by other systems in the LC. This mandatory Core System provides a database, which stores information related to the currently actively offered Services within the Local Cloud.

The rest of this document is organized as follows.
In Section \ref{sec:prior_art}, we reference major prior art capabilities
of the system.
In Section \ref{sec:use}, we the intended usage of the system.
In Section \ref{sec:properties}, we describe fundamental properties
provided by the system.
In Section \ref{sec:delimitations}, we describe delimitations of capabilities
of the system.
In Section \ref{sec:services}, we describe the abstract service
functions consumed or produced by the system.
In Section \ref{sec:security}, we describe the security capabilities
of the system.

\newpage

\subsection{Significant Prior Art}
\label{sec:prior_art}

\color{red}
Describe significant prior art which provides the
foundation for the system - May be omitted for simple services
\color{black}  

\subsection{How This System Is Meant to Be Used}
\label{sec:use}

\color{red}
Describe intended usage of the system. Usage scenarios
shall be described. Preferable a SysML/UML blaock diagram of the
System should be provided. See the SysML profile and library
(github.com/eclipse-arrowhead/profile-library-sysml) for
support on how such block diagram should look like. Suitable tools are
Eclipse Papyrus and MagicDraw.
\color{black}  

\subsection{System functionalities and properties}
\label{sec:properties}

\color{red}
Describe system functionalities and properties like e.g.:
\subsubsection {Functional properties of the system}

\subsubsection {Configuration of the system}
Available parameters

\subsubsection {Data stored by the system}

Brief overview of data stored to achive the functionality of the system. 

\subsubsection {Non functional properties}

\begin{itemize}
  \item security, 
  \item safety, 
  \item energy consumption,
  \item latency
  \item Power saving properties, 
\end{itemize}

\subsubsection {Stateful or stateless}
\begin{itemize} 
\item states preserved, functional and non-functional
\end{itemize}  
\color{black}  


\subsection{Important Delimitations}
\label{sec:delimitations}

\color{red}
Provide delimitations of the provided system. Describe what the system
solve and what i does not solve.
\color{black}  



\newpage

\section{Services}
\label{sec:services}

\color{red}
This section describes consumed and produced service.
In particular, each subsection names a prodiuced or consumed service
indicating the different capabilities and associated interfaces of the
service. Reference to the appropriate SD document shall be made.
\color{black}


\newpage

\section{Security}
\label{sec:security}
\color{red}
\begin{itemize}
\item  If the system can be started in un-secure and/or
Arrowhead secure mode.
\item Handling of Arrowhead compliant and
non-compliant X.509 certificates.
\item Implemented security model shall be described, protocol used, data
protection used, system authentication capability, produced service
authorisation checking, etc.
\end{itemize}
For Arrowhead certificate profile
see github.com/eclipse-arrowhead/documentation
\color{black}




\bibliographystyle{IEEEtran}
\bibliography{bibliography}

\newpage

\section{Revision History}
\subsection{Amendments}

\color{red}
Revision history and Quality assurance as per examples below
\color{black}

\noindent\begin{tabularx}{\textwidth}{| p{1cm} | p{3cm} | p{2cm} | X | p{4cm} |} \hline
\rowcolor{gray!33} No. & Date & Version & Subject of Amendments & Author \\ \hline

1 & 2020-12-05 & \arrowversion & & Tanyi Szvetlin \\ \hline
2 & 2021-07-14 & \arrowversion & Minor updates & Jerker Delsing \\ \hline
3 & 2022-01-12 & \arrowversion & Minor updates & Jerker Delsing \\ \hline
\end{tabularx}

\subsection{Quality Assurance}

\noindent\begin{tabularx}{\textwidth}{| p{1cm} | p{3cm} | p{2cm} | X |} \hline
\rowcolor{gray!33} No. & Date & Version & Approved by \\ \hline

1 & 2022-01-10 & \arrowversion  &  \\ \hline

\end{tabularx}

\end{document}