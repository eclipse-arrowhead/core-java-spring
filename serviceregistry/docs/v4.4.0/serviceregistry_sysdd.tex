\documentclass[a4paper]{arrowhead}

\usepackage[yyyymmdd]{datetime}
\usepackage{etoolbox}
\usepackage[utf8]{inputenc}
\usepackage{multirow}

\renewcommand{\dateseparator}{-}

\setlength{\parskip}{1em}

%% Special references
\newcommand{\fref}[1]{{\textcolor{ArrowheadBlue}{\hyperref[sec:functions:#1]{#1}}}}
\newcommand{\mref}[1]{{\textcolor{ArrowheadPurple}{\hyperref[sec:model:#1]{#1}}}}
\newcommand{\pdef}[1]{{\textcolor{ArrowheadGrey}{#1\label{sec:model:primitives:#1}\label{sec:model:primitives:#1s}\label{sec:model:primitives:#1es}}}}
\newcommand{\pref}[1]{{\textcolor{ArrowheadGrey}{\hyperref[sec:model:primitives:#1]{#1}}}}

\newrobustcmd\fsubsection[3]{
  \addtocounter{subsection}{1}
  \addcontentsline{toc}{subsection}{\protect\numberline{\thesubsection}function \textcolor{ArrowheadBlue}{#1}}
  \renewcommand*{\do}[1]{\rref{##1},\ }
  \Subsection*{
    \thesubsection\quad
    function
    \textcolor{ArrowheadBlue}{#1}
    (\notblank{#2}{\mref{#2}}{})
    \notblank{#3}{: \mref{#3}}{}
  }
  \label{sec:functions:#1}
}
\newrobustcmd\msubsection[2]{
  \addtocounter{subsection}{1}
  \addcontentsline{toc}{subsection}{\protect\numberline{\thesubsection}#1 \textcolor{ArrowheadPurple}{#2}}
  \subsection*{\thesubsection\quad#1 \textcolor{ArrowheadPurple}{#2}}
  \label{sec:model:#2} \label{sec:model:#2s} \label{sec:model:#2es}
}
%%

\begin{document}

%% Arrowhead Document Properties
\ArrowheadTitle{Service Registry Core System}
\ArrowheadType{System Design Description}
\ArrowheadTypeShort{SysDD}
\ArrowheadVersion{4.4.0}
\ArrowheadDate{\today}
\ArrowheadAuthor{Tamás Bordi}
\ArrowheadStatus{RELEASE}
\ArrowheadContact{tbordi@aitia.ai}
\ArrowheadFooter{\href{www.arrowhead.eu}{www.arrowhead.eu}}
\ArrowheadSetup
%%

%% Front Page
\begin{center}
  \vspace*{1cm}
  \huge{\arrowtitle}

  \vspace*{0.2cm}
  \LARGE{\arrowtype}
  \vspace*{1cm}

  %\Large{Service ID: \textit{"\arrowid"}}
  \vspace*{\fill}

  % Front Page Image
  %\includegraphics{figures/TODO}

  \vspace*{1cm}
  \vspace*{\fill}

  % Front Page Abstract
  \begin{abstract}
    This is the template for System Design Description (SysDD document)
    according to the Eclipse Arrowhead documentation structure. 
  \end{abstract}

  \vspace*{1cm}

%   \scriptsize
%   \begin{tabularx}{\textwidth}{l X}
%     \raisebox{-0.5\height}{\includegraphics[width=2cm]{figures/artemis_logo}} & {ARTEMIS Innovation Pilot Project: Arrowhead\newline
%     THEME [SP1-JTI-ARTEMIS-2012-AIPP4 SP1-JTI-ARTEMIS-2012-AIPP6]\newline
%     [Production and Energy System Automation Intelligent-Built environment and urban infrastructure for sustainable and friendly cities]}
%   \end{tabularx}
%   \vspace*{-0.2cm}
 \end{center}

\newpage
%%

%% Table of Contents
\tableofcontents
\newpage
%%

\section{Overview}
\label{sec:overview}
\color{black}
This document describes the Service Registry Core System, which exists to enable service discovery in a Eclipse Arrowhead Local Cloud (LC). In Section \ref{Implementation}, we describe implementation details of the system.

\newpage

\section{Implementation}
\label{Implementation}

\subsection {Implementation language and tools}


\begin{itemize}
\item \emph{Programming Language:} \textbf{Java 11}
\item \emph{Programming Framework:} \textbf{Spring-Boot 2.1.5}
\item \emph{Building Tool:} \textbf{Maven 3.5+}
\item \emph{Database Management System:} \textbf{MySQL 5.7}
\color{red}
\item Stateful or stateless implementation ???? 
\end{itemize} 
\color{black}

\subsection {Functional properties implementation}
\color{red}
  \begin{itemize}
  \item Resources necessary to perform the function operations
  \item Data handled and eventually stored by the system
    \begin{itemize}
    \item Used database
    \item Database structure
    \item Used ontology
    \item Data models
    \end{itemize}
  \item Result provided as ...
  \end{itemize}
\color{black}

\subsection {Non functional properties implementation}
\color{red}
Typical non-functional properties implemented are e.g.
\subsubsection {Security}
\subsubsection {Power management}
\subsubsection {Internal monitoring}
\subsubsection {Configuration}
  \begin{itemize}
  \item Configuration data accepted
  \item Configuration data formats, se blow.
  \item ...  
  \end{itemize}  

For each of the non-functional operations the following details
should be documented 

\begin{itemize}
  \item Resources necessary to perform the non-functional operations
  \item Data handled and eventually stored by the system
    \begin{itemize}
    \item Used database
    \item Database structure
    \item Used ontology
    \item Data models
    \end{itemize}
  \item Result provided as ...
  \end{itemize}
\color{black}




\newpage

\section{Services}
\label{sec:services}

\color{red}
Reference to the appropriate SysD, SD and IDD documents plus a
reference to used library shall be made.

\begin{table*}[ht!]
  \centering
  \caption{References to doumentation for services produced and
    consumed. }
  \label{tab:services}
  \begin{tabular}{|r|r|r|r|}
    \hline
    \rowcolor{gray!33}\textbf{Services produced} & \textbf{SysD ref} & \textbf{SD ref} & \textbf{IDD ref} \\
    \hline
 
    aaa & SysD XXX & SD aaa & IDD aaa \\ \hline
    bbb & SysD XXX & SD bbb & IDD bbb \\ \hline

    &&& \\ \hline

    \rowcolor{gray!33}\textbf{Services consumed} & \textbf{SysD ref} & \textbf{SD ref} & \textbf{IDD ref}\\
    \hline
    zzz & SysD ZZZ & SD zzz & IDD zzz \\ \hline
    yyy & SysD YYY & SD yyy & IDD yyy \\ \hline
    
  \end{tabular}
\end{table*}

\color{black}



\bibliographystyle{IEEEtran}
\bibliography{bibliography}

\newpage

\section{Revision History}
\subsection{Amendments}

\color{red}
Revision history and Quality assurance as per examples below
\color{black}

\noindent\begin{tabularx}{\textwidth}{| p{1cm} | p{3cm} | p{2cm} | X | p{4cm} |} \hline
\rowcolor{gray!33} No. & Date & Version & Subject of Amendments & Author \\ \hline

1 & 2020-12-05 & \arrowversion & & Tanyi Szvetlin \\ \hline
2 & 2021-07-14 & \arrowversion & Minor updates & Jerker Delsing \\ \hline
3 & 2022-01-12 & \arrowversion & Minor updates & Jerker Delsing \\ \hline
\end{tabularx}

\subsection{Quality Assurance}

\noindent\begin{tabularx}{\textwidth}{| p{1cm} | p{3cm} | p{2cm} | X |} \hline
\rowcolor{gray!33} No. & Date & Version & Approved by \\ \hline

1 & 2022-01-10 & \arrowversion  &  \\ \hline

\end{tabularx}

\end{document}