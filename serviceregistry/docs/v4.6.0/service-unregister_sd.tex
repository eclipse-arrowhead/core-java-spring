\documentclass[a4paper]{arrowhead}

\usepackage[yyyymmdd]{datetime}
\usepackage{etoolbox}
\usepackage[utf8]{inputenc}
\usepackage{multirow}

\renewcommand{\dateseparator}{-}

\setlength{\parskip}{1em}

%% Special references
\newcommand{\fref}[1]{{\textcolor{ArrowheadBlue}{\hyperref[sec:functions:#1]{#1}}}}
\newcommand{\mref}[1]{{\textcolor{ArrowheadPurple}{\hyperref[sec:model:#1]{#1}}}}
\newcommand{\pdef}[1]{{\textcolor{ArrowheadGrey}{#1\label{sec:model:primitives:#1}\label{sec:model:primitives:#1s}\label{sec:model:primitives:#1es}}}}
\newcommand{\pref}[1]{{\textcolor{ArrowheadGrey}{\hyperref[sec:model:primitives:#1]{#1}}}}

\newrobustcmd\fsubsection[3]{
  \addtocounter{subsection}{1}
  \addcontentsline{toc}{subsection}{\protect\numberline{\thesubsection}interface \textcolor{ArrowheadBlue}{#1}}
  \renewcommand*{\do}[1]{\rref{##1},\ }
  \subsection*{
    \thesubsection\quad
    interface
    \textcolor{ArrowheadBlue}{#1}
    (\notblank{#2}{\mref{#2}}{})
    \notblank{#3}{: \mref{#3}}{}
  }
  \label{sec:functions:#1}
}
\newrobustcmd\msubsection[2]{
  \addtocounter{subsection}{1}
  \addcontentsline{toc}{subsection}{\protect\numberline{\thesubsection}#1 \textcolor{ArrowheadPurple}{#2}}
  \subsection*{\thesubsection\quad#1 \textcolor{ArrowheadPurple}{#2}}
  \label{sec:model:#2} \label{sec:model:#2s} \label{sec:model:#2es}
}
\newrobustcmd\msubsubsection[3]{
  \addtocounter{subsubsection}{1}
  \addcontentsline{toc}{subsubsection}{\protect\numberline{\thesubsubsection}#1 \textcolor{ArrowheadPurple}{#2}}
  \subsubsection*{\thesubsubsection\quad#1 \textcolor{ArrowheadPurple}{#2}}
  \label{sec:model:#2} \label{sec:model:#2s}
}
%%

\begin{document}

%% Arrowhead Document Properties
\ArrowheadTitle{service-unregister} % XXX = ServiceName 
\ArrowheadServiceID{service-unregister} % ID name of service
\ArrowheadType{Service Description}
\ArrowheadTypeShort{SD}
\ArrowheadVersion{4.6.0} % Arrowhead version X.Y.Z, e..g. 4.4.1
\ArrowheadDate{\today}
\ArrowheadAuthor{Tamás Bordi} % Corresponding author e.g. Jerker Delsing
\ArrowheadStatus{RELEASE} % e..g. RELEASE, RELEASE CONDIDATE, PROTOTYPE
\ArrowheadContact{tbordi@aitia.ai} % Email of corresponding author
\ArrowheadFooter{\href{www.arrowhead.eu}{www.arrowhead.eu}}
\ArrowheadSetup
%%

%% Front Page
\begin{center}
  \vspace*{1cm}
  \huge{\arrowtitle}

  \vspace*{0.2cm}
  \LARGE{\arrowtype}
  \vspace*{1cm}

  %\Large{Service ID: \textit{"\arrowid"}}
  \vspace*{\fill}

  % Front Page Image
  %\includegraphics{figures/TODO}

  \vspace*{1cm}
  \vspace*{\fill}

  % Front Page Abstract
  \begin{abstract}
    This document provides service description for the \textbf{service-unregister} service. 
  \end{abstract}

  \vspace*{1cm}

%   \scriptsize
%   \begin{tabularx}{\textwidth}{l X}
%     \raisebox{-0.5\height}{\includegraphics[width=2cm]{figures/artemis_logo}} & {ARTEMIS Innovation Pilot Project: Arrowhead\newline
%     THEME [SP1-JTI-ARTEMIS-2012-AIPP4 SP1-JTI-ARTEMIS-2012-AIPP6]\newline
%     [Production and Energy System Automation Intelligent-Built environment and urban infrastructure for sustainable and friendly cities]}
%   \end{tabularx}
%   \vspace*{-0.2cm}
 \end{center}

\newpage
%%

%% Table of Contents
\tableofcontents
\newpage
%%

\section{Overview}
\label{sec:overview}
This document describes the \textbf{service-unregister} service, which enables autonomous service unregistration, therefore it is an integral part of the implementation of service discovery requirements in Service Registry Mandatory Core System. Example of this interaction is a system that offers a service through the Service Registry, but for certain reason the service should be terminated. To prevent that this service instance could be offered for other systems, the service provider system is required to remove its service from Service Registry. 

The rest of this document is organized as follows.
In Section \ref{sec:functions}, we describe the abstract message functions provided by the service.
In Section \ref{sec:model}, we end the document by presenting the data types used by the mentioned functions.

\subsection{How This Service Is Meant to Be Used}
The given service provider application system is required to use the  \textbf{service-unregister} service right before, when the service instance is no longer provided. Figure \ref{fig:activity_uml} describes the processing of unregistration data submitted by the application system.

\begin{figure}[h!]
  \centering
  \includegraphics[width=\textwidth]{figures/delete_service_registry_unregister_activity_uml}
  \caption{
    UML activity diagram of service unregistration process.
  }
  \label{fig:activity_uml}
\end{figure}

\subsection{Important Delimitations}
\label{sec:delimitations}

No delimitations.

\subsection{Access policy}
\label{sec:accesspolicy}

Available for anyone within the local cloud, but in case of secure mode service provider is allowed to unregister only its own services. It means that provider system name and system part of certificate common name must match for successful unregistration.

\newpage

\section{Service Interface}
\label{sec:functions}

This section describes the interfaces to the service. The \textbf{service-unregister} service is used to unregister services. In the following, each subsection names an interface, an input type and an output type, in that order.
The input type is named inside parentheses, while the output type is preceded by a colon.
Input and output types are only denoted when accepted or returned, respectively, by the interface in question. All abstract data types named in this section are defined in Section 3.

The following interfaces are available.

\fsubsection{HTTP/TLS/URL}{QueryParams}{Void}

\begin{table}[ht!]
  \centering
  \begin{tabular}{|l|l|l|l|}
    \rowcolor{gray!33} Profile type & Type & Version \\ \hline
    Transfer protocol & HTTP & 1.1 \\ \hline
    Data encryption & TLS & 1.3 \\ \hline
    Encoding & URL & - \\ \hline
    Compression & N/A & - \\ \hline
  \end{tabular}
  \caption{HTTP/TLS/URL communication details.}
  \label{tab:comunication_semantics_profile}
\end{table}

\clearpage

\section{Information Model}
\label{sec:model}

Here, all data objects that can be part of the \textbf{service-unregister} service
provides to the hosting System are listed in alphabetic order.
Note that each subsection, which describes one type of object, begins with the \textit{struct} keyword, which is used to denote a collection of named fields, each with its own data type.
As a complement to the explicitly defined types in this section, there is also a list of implicit primitive types in Section \ref{sec:model:primitives}, which are used to represent things like hashes and identifiers.

\msubsection{struct}{QueryParams}
\label{sec:model:QueryParams}
 
\begin{table}[ht!]
\begin{tabularx}{\textwidth}{| p{3cm} | p{3cm} | p{2cm} | X |} \hline
\rowcolor{gray!33} Field & Type & Mandatory & Description \\ \hline
serviceDefinition &\pref{Name} & yes & Identifier of the service. \\ \hline
providerName & \pref{Name} & yes & Identifier of the provider system. \\ \hline
providerAddress & \pref{Address} & no & Network address. \\ \hline
providerPort &\pref{PortNumber} & yes & Port of the system. \\ \hline
serviceUri & \pref{String} & no & Path of the service on the provider. \\ \hline
\end{tabularx}
\end{table}

\subsection{Primitives}
\label{sec:model:primitives}

Types and structures mentioned throughout this document that are assumed to be available to implementations of this service.
The concrete interpretations of each of these types and structures must be provided by any IDD document claiming to implement this service.


\begin{table}[ht!]
\begin{tabularx}{\textwidth}{| p{3cm} | X |} \hline
\rowcolor{gray!33} Type & Description \\ \hline
\pdef{Address}          & A string representation of the address. \\ \hline
\pdef{Name}             & A string identifier that is intended to be both human and machine-readable. \\ \hline
\pdef{PortNumber}       & Decimal number in the range of 0-65535. \\ \hline
\pdef{String}           & A chain of characters. \\ \hline
\end{tabularx}
\end{table}

\newpage

\bibliographystyle{IEEEtran}
\bibliography{bibliography}

\newpage

\section{Revision History}
\subsection{Amendments}

\noindent\begin{tabularx}{\textwidth}{| p{1cm} | p{3cm} | p{2cm} | X | p{4cm} |} \hline
\rowcolor{gray!33} No. & Date & Version & Subject of Amendments & Author \\ \hline

1 & YYYY-MM-DD & \arrowversion & & Xxx Yyy \\ \hline
\end{tabularx}

\subsection{Quality Assurance}

\noindent\begin{tabularx}{\textwidth}{| p{1cm} | p{3cm} | p{2cm} | X |} \hline
\rowcolor{gray!33} No. & Date & Version & Approved by \\ \hline

1 & YYYY-MM-DD & \arrowversion  &  \\ \hline

\end{tabularx}

\end{document}