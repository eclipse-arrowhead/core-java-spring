\documentclass[a4paper]{arrowhead}

\usepackage[yyyymmdd]{datetime}
\usepackage{etoolbox}
\usepackage[utf8]{inputenc}
\usepackage{multirow}

\renewcommand{\dateseparator}{-}

\setlength{\parskip}{1em}

%% Special references
\newcommand{\fref}[1]{{\textcolor{ArrowheadBlue}{\hyperref[sec:functions:#1]{#1}}}}
\newcommand{\mref}[1]{{\textcolor{ArrowheadPurple}{\hyperref[sec:model:#1]{#1}}}}
\newcommand{\pdef}[1]{{\textcolor{ArrowheadGrey}{#1\label{sec:model:primitives:#1}\label{sec:model:primitives:#1s}\label{sec:model:primitives:#1es}}}}
\newcommand{\pref}[1]{{\textcolor{ArrowheadGrey}{\hyperref[sec:model:primitives:#1]{#1}}}}

\newrobustcmd\fsubsection[3]{
  \addtocounter{subsection}{1}
  \addcontentsline{toc}{subsection}{\protect\numberline{\thesubsection}interface \textcolor{ArrowheadBlue}{#1}}
  \renewcommand*{\do}[1]{\rref{##1},\ }
  \subsection*{
    \thesubsection\quad
    interface
    \textcolor{ArrowheadBlue}{#1}
    (\notblank{#2}{\mref{#2}}{})
    \notblank{#3}{: \mref{#3}}{}
  }
  \label{sec:functions:#1}
}
\newrobustcmd\msubsection[2]{
  \addtocounter{subsection}{1}
  \addcontentsline{toc}{subsection}{\protect\numberline{\thesubsection}#1 \textcolor{ArrowheadPurple}{#2}}
  \subsection*{\thesubsection\quad#1 \textcolor{ArrowheadPurple}{#2}}
  \label{sec:model:#2} \label{sec:model:#2s} \label{sec:model:#2es}
}
\newrobustcmd\msubsubsection[3]{
  \addtocounter{subsubsection}{1}
  \addcontentsline{toc}{subsubsection}{\protect\numberline{\thesubsubsection}#1 \textcolor{ArrowheadPurple}{#2}}
  \subsubsection*{\thesubsubsection\quad#1 \textcolor{ArrowheadPurple}{#2}}
  \label{sec:model:#2} \label{sec:model:#2s}
}
%%

\begin{document}

%% Arrowhead Document Properties
\ArrowheadTitle{service-register} % XXX = ServiceName 
\ArrowheadServiceID{service-register} % ID name of service
\ArrowheadType{Service Description}
\ArrowheadTypeShort{SD}
\ArrowheadVersion{4.6.0} % Arrowhead version X.Y.Z, e..g. 4.4.1
\ArrowheadDate{\today}
\ArrowheadAuthor{Tamás Bordi} % Corresponding author e.g. Jerker Delsing
\ArrowheadStatus{RELEASE} % e..g. RELEASE, RELEASE CONDIDATE, PROTOTYPE
\ArrowheadContact{tbordi@aitia.ai} % Email of corresponding author
\ArrowheadFooter{\href{www.arrowhead.eu}{www.arrowhead.eu}}
\ArrowheadSetup
%%

%% Front Page
\begin{center}
  \vspace*{1cm}
  \huge{\arrowtitle}

  \vspace*{0.2cm}
  \LARGE{\arrowtype}
  \vspace*{1cm}

  %\Large{Service ID: \textit{"\arrowid"}}
  \vspace*{\fill}

  % Front Page Image
  %\includegraphics{figures/TODO}

  \vspace*{1cm}
  \vspace*{\fill}

  % Front Page Abstract
  \begin{abstract}
    This document provides service description for the \textbf{service-register} service. 
  \end{abstract}

  \vspace*{1cm}

%   \scriptsize
%   \begin{tabularx}{\textwidth}{l X}
%     \raisebox{-0.5\height}{\includegraphics[width=2cm]{figures/artemis_logo}} & {ARTEMIS Innovation Pilot Project: Arrowhead\newline
%     THEME [SP1-JTI-ARTEMIS-2012-AIPP4 SP1-JTI-ARTEMIS-2012-AIPP6]\newline
%     [Production and Energy System Automation Intelligent-Built environment and urban infrastructure for sustainable and friendly cities]}
%   \end{tabularx}
%   \vspace*{-0.2cm}
 \end{center}

\newpage
%%

%% Table of Contents
\tableofcontents
\newpage
%%

\section{Overview}
\label{sec:overview}
This document describes the \textbf{service-register} service, which enables autonomous service registration, therefore it is an integral part of the implementation of service discovery requirements in Service Registry Mandatory Core System. Example of this interaction is a system that has the capability to provide some kind of service. To enable other systems to use, to consume it, this service needs to be offered through the ServiceRegistry.

The rest of this document is organized as follows.
In Section \ref{sec:functions}, we describe the abstract message functions provided by the service.
In Section \ref{sec:model}, we end the document by presenting the data types used by the mentioned functions.

\subsection{How This Service Is Meant to Be Used}
The given service provider application system is required to use the  \textbf{service-register} service at its startup in order the offered services are being discoverable for the possible consumer application systems within the local cloud. Figure \ref{fig:activity_uml} describes the processing of registration data submitted by the application system.

\begin{figure}[h!]
  \centering
  \includegraphics[width=12.5cm]{figures/post_service_registry_register_activity_uml}
  \caption{
    UML activity diagram of service registration process.
  }
  \label{fig:activity_uml}
\end{figure}

\newpage

\subsection{Important Delimitations}
\label{sec:delimitations}

The registration data must meet the following criteria:

\begin{itemize}
    \item Service definition can contain maximum 63 character of letters (english alphabet), numbers and dash (-), and has to start with a letter (also cannot ends with dash).
    \item System name can't be what is reserved for core systems.
    \item System name can contain maximum 63 character of letters (english alphabet), numbers and dash (-), and have to start with a letter (also cannot end with dash).
    \item Interface name has to follow the \texttt{Protocol-SecurityType-MimeType} format.
    \item Security types could be only \texttt{NOT\_SECURE}, \texttt{CERTIFICATE} or \texttt{TOKEN} .
\end{itemize}

\subsection{Access policy}
\label{sec:accesspolicy}

Available for anyone within the local cloud, but in case of secure mode service provider is allowed to register only its own services. It means that provider system name and system part of certificate common name must match for successful registration.

\textit{Exception:} Translator Supporting Core Sytem is allowed to register other services too.

\newpage

\section{Service Interface}
\label{sec:functions}

This section describes the interfaces to the service. The \textbf{service-register} service is used to register services. A service could contain various metadata as well as a physical endpoint. The various parameters are representing the necessary system and service input information.
In particular, each subsection names an interface, an input type and an output type, in that order.
The input type is named inside parentheses, while the output type is preceded by a colon.
Input and output types are only denoted when accepted or returned, respectively, by the interface in question. All abstract data types named in this section are defined in Section 3.

The following interfaces are available.

\fsubsection{HTTP/TLS/JSON}{ServiceRegistryRequest}{ServiceRegistryResponse}

\begin{table}[ht!]
  \centering
  \begin{tabular}{|l|l|l|l|}
    \rowcolor{gray!33} Profile type & Type & Version \\ \hline
    Transfer protocol & HTTP & 1.1 \\ \hline
    Data encryption & TLS & 1.3 \\ \hline
    Encoding & JSON & RFC 8259 \cite{rfc8259} \\ \hline
    Compression & N/A & - \\ \hline
  \end{tabular}
  \caption{HTTP/TLS/JSON communication details.}
  \label{tab:comunication_semantics_profile}
\end{table}

\clearpage

\section{Information Model}
\label{sec:model}

Here, all data objects that can be part of the \textbf{service-register} service
provides to the hosting System are listed in alphabetic order.
Note that each subsection, which describes one type of object, begins with the \textit{struct} keyword, which is used to denote a collection of named fields, each with its own data type.
As a complement to the explicitly defined types in this section, there is also a list of implicit primitive types in Section \ref{sec:model:primitives}, which are used to represent things like hashes and identifiers.

\msubsection{struct}{ServiceRegistryRequest}
 
\begin{table}[ht!]
\begin{tabularx}{\textwidth}{| p{3cm} | p{3cm} | p{2cm} | X |} \hline
\rowcolor{gray!33} Field & Type & Mandatory & Description \\ \hline
endOfValidity & \pref{DateTime} & no & Service is available until this UTC timestamp. \\ \hline
interfaces & \pref{List}$<$\pref{Interface}$>$ & yes & List of interfaces the service supports. \\ \hline
metadata &\hyperref[sec:model:Metadata]{Metadata} & no & Additional information about the system. \\ \hline
providerSystem & \hyperref[sec:model:SystemDescriptor]{SystemDescriptor} & yes & Descriptor of the provider system. \\ \hline
secure &\pref{SecureType} & yes & Type of security the service uses. \\ \hline
serviceDefinition &\pref{Name} & yes & Identifier of the service. \\ \hline
serviceUri & \pref{String} & no & Path of the service on the provider. \\ \hline
version &\pref{Version} & yes & Version of the service. \\ \hline
\end{tabularx}
\end{table}

\msubsection{struct}{Metadata}
\label{sec:model:Metadata}

An \pref{Object} which maps \pref{String} key-value pairs.

\msubsection{struct}{SystemDescriptor}
\label{sec:model:SystemDescriptor}

\begin{table}[ht!]
\begin{tabularx}{\textwidth}{| p{4cm} | p{4cm} | p{2cm} | X |} \hline
\rowcolor{gray!33} Field & Type & Mandatory & Description \\ \hline

address &\pref{Address} & yes & Network address of the system. \\ \hline
authenticationInfo &\pref{String} & no & X.509 public key of the system. \\ \hline
metadata &\hyperref[sec:model:Metadata]{Metadata} & no & Additional information about the system. \\ \hline
port &\pref{PortNumber} & yes & Port of the system. \\ \hline
systemName &\pref{Name} & yes & Name of the system. \\ \hline
\end{tabularx}
\end{table}

\clearpage

\msubsection{struct}{ServiceRegistryResponse}
 
\begin{table}[ht!]
\begin{tabularx}{\textwidth}{| p{4.25cm} | p{4cm} | X |} \hline
\rowcolor{gray!33} Field & Type      & Description \\ \hline
createdAt & \pref{DateTime} & Service instance record was created at this UTC timestamp. \\ \hline
endOfValidity & \pref{DateTime} & Service is available until this UTC timestamp. \\ \hline
id & \pref{Number} & Identifier of the service instance \\ \hline
interfaces & \pref{List}$<$\hyperref[sec:model:InterfaceRecord]{InterfaceRecord}$>$     & List of interfaces the service supports. \\ \hline
metadata & \hyperref[sec:model:Metadata]{Metadata} & Additional information about the system. \\ \hline
providerSystem & \hyperref[sec:model:SystemRecord]{SystemRecord} & Descriptor of the provider system record. \\ \hline
secure &\pref{SecureType}  & Type of security the service uses. \\ \hline
serviceDefinition & \hyperref[sec:model:ServiceDefinitionRecord]{ServiceDefinitionRecord} & Descriptor of the service definition record. \\ \hline
serviceUri & \pref{String} & Path of the service on the provider. \\ \hline
updatedAt & \pref{DateTime} & Service instance record was modified at this UTC timestamp. \\ \hline
version &\pref{Version}     & Version of the service. \\ \hline
\end{tabularx}
\end{table}

\msubsection{struct}{InterfaceRecord}
\label{sec:model:InterfaceRecord}

\begin{table}[ht!]
\begin{tabularx}{\textwidth}{| p{4.25cm} | p{3.5cm} | X |} \hline
\rowcolor{gray!33} Field & Type      & Description \\ \hline
createdAt & \pref{DateTime} & Interface instance record was created at this UTC time\-stamp. \\ \hline
id & \pref{Number} & Identifier of the interface instance \\ \hline
interfaceName & \pref{Interface}  & Specified name of the interface. \\ \hline
updatedAt & \pref{DateTime} & Interface instance record was modified at this UTC time\-stamp. \\ \hline
\end{tabularx}
\end{table}

\msubsection{struct}{SystemRecord}
\label{sec:model:SystemRecord}

\begin{table}[ht!]
\begin{tabularx}{\textwidth}{| p{4.25cm} | p{3.5cm} | X |} \hline
\rowcolor{gray!33} Field & Type & Description \\ \hline

address &\pref{Address} & Network address of the system. \\ \hline
authenticationInfo &\pref{String} & X.509 public key of the system. \\ \hline
createdAt & \pref{DateTime} & System instance record was created at this UTC time\-stamp. \\ \hline
id & \pref{Number} & Identifier of the system instance. \\ \hline
metadata &\hyperref[sec:model:Metadata]{Metadata} & Additional information about the system. \\ \hline
port &\pref{PortNumber} & Port of the system. \\ \hline
systemName &\pref{Name} & Name of the system. \\ \hline
updatedAt & \pref{DateTime} & System instance record was modified at this UTC time\-stamp. \\ \hline
\end{tabularx}
\end{table}

\clearpage

\msubsection{struct}{ServiceDefinitionRecord}
\label{sec:model:ServiceDefinitionRecord}

\begin{table}[ht!]
\begin{tabularx}{\textwidth}{| p{4.25cm} | p{3.5cm} | X |} \hline
\rowcolor{gray!33} Field & Type      & Description \\ \hline
createdAt & \pref{DateTime} & Service definition instance record was created at this UTC timestamp. \\ \hline
id & \pref{Number} & Identifier of the service definition instance \\ \hline
serviceDefinition &\pref{Name}  & Name of the service definition. \\ \hline
updatedAt & \pref{DateTime} & Service definition instance record was modified at this UTC timestamp. \\ \hline
\end{tabularx}
\end{table}

\subsection{Primitives}
\label{sec:model:primitives}

Types and structures mentioned throughout this document that are assumed to be available to implementations of this service.
The concrete interpretations of each of these types and structures must be provided by any IDD document claiming to implement this service.


\begin{table}[ht!]
\begin{tabularx}{\textwidth}{| p{3cm} | X |} \hline
\rowcolor{gray!33} Type & Description \\ \hline
\pdef{Address}          & A string representation of the address \\ \hline
\pdef{Boolean}          & One out of \texttt{true} or \texttt{false}. \\ \hline
\pdef{DateTime}         & Pinpoints a specific moment in time. \\ \hline
\pdef{Interface}        & Any suitable type chosen by the implementor of service \\ \hline
\pdef{List}$<$A$>$      & An \textit{array} of a known number of items, each having type A. \\ \hline
\pdef{Name}             & A string identifier that is intended to be both human and machine-readable. \\ \hline
\pdef{Number}           & Decimal number \\ \hline
\pdef{Object}           & Set of primitives and possible further objects. \\ \hline
\pdef{PortNumber}       & A \pref{Number} between 0 and 65535. \\ \hline
\pdef{SecureType}       & Any suitable type chosen by the implementor of service \\ \hline
\pdef{String}           & A chain of characters. \\ \hline
\pdef{Version}          & Specifies a service version. \\ \hline
\end{tabularx}
\end{table}

\newpage

\bibliographystyle{IEEEtran}
\bibliography{bibliography}

\newpage

\section{Revision History}
\subsection{Amendments}

\noindent\begin{tabularx}{\textwidth}{| p{1cm} | p{3cm} | p{2cm} | X | p{4cm} |} \hline
\rowcolor{gray!33} No. & Date & Version & Subject of Amendments & Author \\ \hline

1 & YYYY-MM-DD & \arrowversion & & Xxx Yyy \\ \hline
\end{tabularx}

\subsection{Quality Assurance}

\noindent\begin{tabularx}{\textwidth}{| p{1cm} | p{3cm} | p{2cm} | X |} \hline
\rowcolor{gray!33} No. & Date & Version & Approved by \\ \hline

1 & YYYY-MM-DD & \arrowversion  &  \\ \hline

\end{tabularx}

\end{document}